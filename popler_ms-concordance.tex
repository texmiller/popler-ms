\Sconcordance{concordance:popler_ms.tex:popler_ms.Rnw:%
1 37 1 50 0 1 10 10 1 1 8 37 1 1 30 36 1 4 0 5 1 4 0 220 1}
